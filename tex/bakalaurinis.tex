% vim: tw=80 fo=aw2tq:
\documentclass{VUMIFInfBakalaurinis}
\usepackage{algorithmicx}
\usepackage{algorithm}
\usepackage{algpseudocode}
\usepackage{amsfonts}
\usepackage{amsmath}
\usepackage{bm}
\usepackage{caption}
\usepackage{color}
\usepackage{float}
\usepackage{graphicx}

\usepackage{pgfplots}
\usepackage{pgfplotstable}
\pgfplotsset{compat=1.7}
\usepackage{tikz}

% \usepackage{hyperref}  % Nuorodų aktyvavimas
\usepackage{listings}
\usepackage{subfig}
\usepackage{url}
\usepackage{wrapfig}

% Titulinio aprašas
\university{Vilniaus universitetas}
\faculty{Matematikos ir informatikos fakultetas}
\institute{Informatikos institutas}
\department{Informatikos katedra}
\papertype{Baigiamasis bakalauro darbas}
\title{Naujausių automatizuoto slaptažodžių parinkimo metodų palyginimas}
\titleineng{Comparison of the latest automated password guessing techniques}
\status{4 kurso 1 grupės studentas}
\author{Gediminas Valys}
% \secondauthor{Vardonis Pavardonis}   % Pridėti antrą autorių
\supervisor{prof. dr. Igoris Belovas}
\reviewer{doc. dr. Vardauskas Pavardauskas}
% \date{Vilnius \\ \the\year}
\date{Vilnius \\ 2022}

\setmainfont{Palemonas}
\bibliography{bibliografija} 

\begin{document}
\maketitle

\tableofcontents

\sectionnonum{Sąvokų apibrėžimai}
Sutartinių ženklų, simbolių, vienetų ir terminų sutrumpinimų sąrašas (jeigu
ženklų, simbolių, vienetų ir terminų bendras skaičius didesnis nei 10 ir
kiekvienas iš jų tekste kartojasi daugiau nei 3 kartus).

\section{Santrauka}
Santrauka lietuviškai.

\section{Įvadas}
% Kokie darbo tikslai
% Kas buvo sukurta, kokios problemos išspręstos, kokie tyrimai vykdyti
% Kokie rezultatai pasiekti
% Likusių dalių struktūra
Vartotojo autentifikacija slaptažodžiu yra vienas iš pagrindinių kibernetinio 
saugumo metodų. Kai yra poreikis nustatyti vartotojo slaptažodį (pvz. 
kriminalistiniuose, ekspertiniuose tyrimuose, siekiant nustatyti kaltę), 
dabartiniai populiariausi metodai su palyginti mažais kompiuteriniais resursais 
gali nustatyti slaptažodį iki 6 simbolių ilgio, \textquote{ASCII} (angl. 
\textquote{American Standard Code for Information Interchange}) koduotėje, 
tačiau ilgesniams slaptažodžiams kompiuterinių išteklių ir laiko reikalavimai 
žymiai padidėja ir darbas netenka prasmės. Naujausi slaptažodžių parinkimo 
metodai, pagrįsti tikimybine, bekontekstine gramatika (angl. 
\textquote{Probabilistic context-free grammar}) ir mašininio mokymosi metodais 
gali identifikuoti slaptažodžių struktūras bei ypatybes, ir parinkti ilgesnius 
nei 6 simbolių slaptažodžius su palyginti mažais kompiuteriniais resursais. 
Šiame darbe siekiama palyginti du tokius automatizuotus slaptažodžio parinkimo 
metodus ir įvertinti jų parenkamus slaptažodžius atliekant eksperimentų seriją 
su nutekintomis slaptažodžių duomenų bazėmis.

\section{Literatūros apžvalga}
% Ne tik pasakyti ką atliko kiti autoriai, bet palyginti jų darbus, pateikti tų 
% darbų pliusus ir minusus. Nuo literatūros analizės turi priklausyti ir 
% pagrindinėje dalyje pasirinkti metodai
\subsection{\textquote{PCFG} Algoritmas}
\textquote{PCFG} (angl. \textquote{Probabilistic Context Free Grammar}) - 
tikimybinis gramatikos taisyklių rinkinys, nurodantis, kaip kažkurį tai 
simbolį-reikšmę galima transformuoti į kitą simbolį. Pagrindinis algoritmo 
tikslas yra identifikuoti žmonių slaptažodžių kūrimo įpročius, ir, remiantis jų 
įpročių, slaptažodžių požymiais ir ypatybėmis, parinkti naujus slaptažodžius.

Slaptažodžių parinkimo strategijoje, \textquote{PCFG} algoritmas peržiūri 
slaptažodžių aibę, kad nustatytų, kokias dažnai pasikartojančias, labiausiai 
tikėtinas transformacijas galėtų atlikti su slaptažodžiais, ir parenka naujus 
slaptažodžius su tomis transformacijomis.

Kiekvienam slaptažodžiui iš apmokymo duomenų rinkinio algoritmas pirmiausia 
išveda gramatiką, t.y. simbolių grupės yra sugrupuojamos pagal tipą: skaičiai, 
raidės, specialūs simboliai, ir užfiksuojamas konkrečių simbolių sekų bei pačios 
gramatikos pasikartojimų skaičius. Kai visi slaptažodžiai yra apdoroti į jų 
atitinkamas gramatikas, paskaičiuojami gramatikų ir simbolių grupių 
pasikartojimo dažniai. Minėti dažniai yra naudojami parenkant naujus 
slaptažodžius - pirmiausia parenkami slaptažodžiai, kurių gramatika ir 
gramatikoje esančių simbolių grupių dažniai yra didžiausi, tokiu būdu labiausiai 
slaptažodžiai su aukščiausia tikimybe yra parenkami pirmi. Taip pat garantuojama 
algoritmo pabaiga - arba pasibaigia gramatikos, arba pasibaigia unikalios 
simbolių grupės - ir nauji slaptažodžiai nebeparenkami. Parenkamų slaptažodžių 
bei gramatikų ir simbolių grupių kiekiai priklauso nuo apmokymo duomenų rinkinio 
dydžio - yra svarbūs pasikartojantys duomenys, kurie naudojami tikimybių 
skaičiavimams.

Viena gramatika \textquote{PCFG} algoritme gali būti panaudota kelis kartus 
parenkant slaptažodžius. \textquote{PCFG} algoritme yra išskiriami du tipai 
slaptažodžių - preliminarūs (angl. \textquote{Pre-terminal}) ir galutiniai 
(angl. \textquote{Terminal}). Kiekvienas preliminarus slaptažodis yra parenkamas 
atsižvelgiant į jo ašies reikšmę, t.y. simbolių grupės, kuri buvo 
pakeista/transformuota parenkant šį slaptažodį, indeksas (pirminė ašies reikšmė 
parenkant pirmuosius preliminarius slaptažodžius yra lygi nuliui). Galutinių 
slaptažodžių parinkimo ir preliminarių slaptažodžių transformavimo taisyklės yra 
tokios:
\begin{enumerate}
  \item preliminariame slaptažodyje simbolių grupė gali būti keičiama tik jei 
    preliminaraus slaptažodžio ašies vertė yra mažesnė už grupės indeksą 
    slaptažodžio gramatikoje;
  \item preliminaraus slaptažodžio nauja ašis yra slaptažodyje pakeistos 
    simbolių grupės indeksas, skaičiuojant nuo 0;
  \item iš preliminaraus slaptažodžio yra parenkamas galutinis slaptažodis kai 
    yra pakeičiama viena ir tik viena simbolių grupė.
\end{enumerate}

Vaduvaujantis aukščiau minėtomis taisyklėmis yra parenkami nauji galutiniai ir 
preliminarūs slaptažodžiai su transformuotomis simbolių grupėmis. Galutinis 
slaptažodis gali būti taikomas vartotojo kokio nors slaptažodžio atkūrimui iš 
maišos funkcijos išvestos kontrolinės sumos (pvz. MD5, SHA-1 ar kitos maišos 
funkcijos). Nauji preliminarūs slaptažodžiai yra grąžinami algoritmui, tolesnėms 
transformacijoms, kol iš preliminaraus slaptažodžio gramatikos bei atitinkamų 
simbolių grupių galima sujungti/transformuoti naujas reikšmes.

\subsection{\textquote{PassGAN} Algoritmas}
\subsubsection{Apie generatyvinius konkurencinius tinklus}
Generatiniai konkurenciniai tinklai (angl. \textquote{Generative Adversarial 
Networks}, \textquote{GAN}) yra generatyvinis modeliavimas naudojant gilaus 
mašininio mokimosi metodus. Generatyvinis modeliavimas yra mašininio mokymosi 
metodas be mokytojo, kuriame modelis yra apmokamas generuoti tokias reikšmes, 
kurios galėtų tikėtinai būti iš apmokymo duomenų rinkinio. Generatyviniame 
konkurenciniame tinkle yra dvi dalys - generavimo modelis (angl. 
\textquote{Generator model}), kuris yra apmokamas generuoti naujas reikšmes, ir 
diskriminatoriaus modelis (angl. \textquote{Discriminator model}), kuris 
klasifikuoja generuojamas reikšmes kaip tikras arba netikras. Generativinių 
konkurencinių tinklų pagrindas yra teoriškas žaidybinis scenarijus, kuriame 
minėti du modeliai konkuruoja ir varžosi - vienas modelis generuoja reikšmes, 
kitas bando atskirti jas nuo generatoriaus ir apmokymo duomenų 
\cite{Goodfellow-et-al-2016}. Generavimo modelis yra apmokamas tol, kol 
generuoja tokias įtikinamas reikšmes, kurias diskriminatoriaus modelis negali 
atskirti nuo realių (apmokymo) reikšmių.

Pagrindinis skirtumas tarp diskriminuojančio ir generatyvinio modeliavimo yra 
tai, diskriminacinis modelis klasifikuoja - turi būti priimtas sprendimas, 
kuriai klasei priklauso tam tikra reikšmė, o generatyvinis - generuoja arba 
sukuria naujas reikšmes pagal pradinius duomenis, kurie galėtų tikėtinai būti 
pradiniame duomenų rinkinyje. Kitaip tariant, generatyvinis modelis gali būti 
apmokytas pradinių duomenų pasiskirstymo ir gali generuoti naujas reikšmes, 
kurios galėtų tikėtinai būti priskirtos prie pradinių duomenų \cite{Bishop07}.

Generavimo modeliui yra pateikiama kaip įvestis nustatyto ilgio vektorius iš 
atsitiktinai paskirstytos duomenų aibės (angl. \textquote{Gaussian 
distribution}). Apmokymo metu tam tikriems (paprastai nematomiems) požymiams iš 
šios duomenų aibės yra priskiriama svarba, t.y. generuojamos reikšmės artėja 
prie apmokymo duomenų aibės - generatorius sugeneruoja grupę reikšmių, kurios, 
kartu su apmokymo duomenimis, yra perduodami diskriminatoriui klasifikuoti kaip 
tikros arba netikros. Tokiu būdu diskriminatoriaus parametrai yra tobulinami 
tiksliau klasifikuoti tikras/netikras reikšmes, o generatorius - kiek 
sugeneruotos reikšmės galėjo apgauti/įtikinti diskriminatorių (diskriminatoriaus 
reikšmės, nustatytos kaip tikros arba netikros, atskleidžia reikalingą 
informaciją apie apmokymo duomenų aibę generatoriui, kuris gali ją naudoti 
tiksliau modeliuoti apmokymo duomenų pasiskirstymą). Po modelio apmokymo, minėta 
duomenų aibė turėtų atitikti apmokymo duomenų aibę, t.y. diskriminatorius negali 
su didesniu nei apie 50 \% tikslumu (pasitikėjimu, arba angl. 
\textquote{Confidence}) atskirti apmokymo duomenų aibės reikšmes nuo generuojamų 
reikšmių.
\subsubsection{\textquote{PassGAN} metodo požymiai}
\textquote{PassGAN} metodui, kuriame generatyvinis konkurencinis tinklas yra 
pritaikytas slaptažodžių generavimui, yra reikalingi palyginti 


\cite{DBLP:journals/corr/abs-1709-00440}

\section{Pagrindinė tiriamoji dalis}
Pagrindinėje tiriamojoje dalyje aptariama ir pagrindžiama tyrimo metodika;
pagal atitinkamas darbo dalis, nuosekliai, panaudojant lyginamosios analizės,
klasifikacijos, sisteminimo metodus bei apibendrinimus, dėstoma sukaupta ir
išanalizuota medžiaga. 

\subsection{Etika}
Šis darbas buvo atliktas pagal Europos elgesio kodeksą mokslinių tyrimų etikos 
klausimais
(angl. \textquote{European Code of Conduct for Research Integrity}), paruoštą 
\textquote{European Federation of Academies of Sciences and Humanities} (ALLEA) 
organizacijos. Kadangi nutekintos duomenų bazės yra sudarytos iš galimų 
vartotojų asmeninės informacijos (pvz. vardas, pavardė, elektroninis pašto 
adresas), visa informacija buvo saugiai laikoma ir tvarkoma. Asmens duomenų 
atskleidimo galimybės buvo mažinamos įgyvendinant griežtas saugumo priemones, ir 
nutekinti slaptažodžiai nebuvo testuojami su tikromis internetinių paslaugų 
vartotojų paskyromis. Darbe atskleidžiami tik patys dažniausi slaptažodžiai, jų 
sudėtingumas ir struktūra. Slaptažodžiai, kurie galėtų būti naudojami 
identifikuoti vartotojus, nebuvo atskleisti. Vartotojų asmeninė informacija yra 
nepateikta.

\subsection{Nutekintų slaptažodžių duomenų bazių analizė}
Šiame darbe naudojamos trys nutekintų slaptažodžių duomenų bazės - A, B ir C. 
Duomenų bazę (rinkinį) C sudaro slaptažodžių SHA-1 maišos funkcijos kontrolinės 
sumos. Dešimt dažniausiai pasikartojančių slaptažodžių duomenų bazėje A yra 
pateikti \ref{10-dazn-slapt-a} lentelėje, o duomenų bazėje B - 
\ref{10-dazn-slapt-b} lentelėje.

% $ sort DB.txt | uniq -c | sort -nr | head -n 10
\begin{table}[hb]
\centering
\begin{tabular}{|c|c|}
  \hline
  Kiekis & Slaptažodis \\
  \hline
  661 & 123456 \\
  363 & aaaaaa \\
  313 & 123456789 \\
  187 & qwerty \\
  125 & vasara \\
  122 & pavasaris \\
  122 & kaktusas \\
  119 & slaptazodis \\
  116 & katinas \\
  107 & 12345678 \\
  \hline
\end{tabular}
\caption{Duomenų rinkinio A 10 dažniausiai pasikartojančių slaptažodžių}
\label{10-dazn-slapt-a}
\end{table}

\ref{10-dazn-slapt-a} lentelėje pateikti slaptažodžiai daugiausia sudaryti iš 
bendrinių daiktavardžių, paprastų skaičių ir raidžių sekų.

\begin{table}[ht]
\centering
\begin{tabular}{|c|c|}
  \hline
  Kiekis & Slaptažodis \\
  \hline
  2252 & x4ivygA51F \\
  1341 & 123456789 \\
  1302 & h54rsjrF5J46788998 \\
  993 & 64t3zaWonZ \\
  917 & 2u8Qujf5eE \\
  896 & slaptazodis \\
  893 & 88ggOx8ouG \\
  828 & lopas123 \\
  823 & 5ogs6d4QnbA \\
  761 & Y52flzfq8V \\
  \hline
\end{tabular}
\caption{Duomenų rinkinio B 10 dažniausiai pasikartojančių slaptažodžių}
\label{10-dazn-slapt-b}
\end{table}

Kaip yra pavaizduota \ref{10-dazn-slapt-b} lentelėje, kitaip nei 
\ref{10-dazn-slapt-a} lentelėje, dauguma slaptažodžių yra sudėtingesni skaičių 
ir simbolių sekų mišiniai, kuriuose bendriniai daiktavardžiai yra retesni.

% https://shantoroy.com/latex/line-graph-pgfplots/
\begin{tikzpicture}
\begin{axis}[
  xlabel=Sugeneruotų slaptažodžių kiekis, vnt.,
  ylabel=Sutampančių slaptažodžių kiekis, vnt.,
  width=10cm,height=7cm,
  legend style={at={(0.0,.91)},anchor=west}
]

% Add values and attributes for the first plot
\addplot[color=red,mark=*] coordinates {
  (10^6, 6)
  (10^7, 7)
  (10^8, 8)
  (10^9, 9)
  (10^10, 10)
};

% Add values and attributes for the second plot
\addplot[color=blue,mark=*] coordinates {
  (10^6, 8)
  (10^7, 9)
  (10^8, 10)
  (10^9, 11)
  (10^10, 12)
};

\legend{Case 1,Case 2}
\end{axis}
\end{tikzpicture}


\subsection{Eksperimentuose taikoma strategija}
Su pasirinktu automatizuoto slaptažodžių parinkimo metodu, apmokyti modelį su 80 
\% duomenų iš duomenų rinkinių A ir B, bei sumaišyto A ir B duomenų rinkinio. 
Apmokytu modeliu sugeneruoti tarp $10^{6}$ ir $10^{10}$ slaptažodžių, ir, 
naudojantis specialia programine įranga \textquote{Hashcat}, nustatyti, kiek 
slaptažodžių iš sugeneruotų slaptažodžių aibės sutampa su duomenų rinkinyje C 
slaptažodžiais.

\subsection{\textquote{PCFG} eksperimentų serija}

\subsection{\textquote{PassGAN} eksperimentų serija}

\sectionnonum{Išvados}
Išvadose ir pasiūlymuose, nekartojant atskirų dalių apibendrinimų,
suformuluojamos svarbiausios darbo išvados, rekomendacijos bei pasiūlymai.

\sectionnonum{Conclusions}
Šiame skyriuje pateikiamos išvados (reziume) anglų kalba.


\printbibliography[heading=bibintoc]

\appendix  % Priedai
% Prieduose gali būti pateikiama pagalbinė, ypač darbo autoriaus savarankiškai
% parengta, medžiaga. Savarankiški priedai gali būti pateikiami kompiuterio
% diskelyje ar kompaktiniame diske. Priedai taip pat vadinami ir numeruojami.
% Tekstas su priedais siejamas nuorodomis (pvz.: \ref{img:mlp}).

% \section{Niauroninio tinklo struktūra}
% \begin{figure}[H]
%     \centering
%     \includegraphics[scale=0.5]{img/MLP}
%     \caption{Paveikslėlio pavyzdys}
%     \label{img:mlp}
% \end{figure}

\section{Eksperimentinio palyginimo rezultatai}

\end{document}
