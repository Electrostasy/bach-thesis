% vim: tw=80 fo=aw2tq ts=2 shiftwidth=2 et:
\documentclass{VUMIFInfBakalaurinis}
\usepackage{algorithm}
\usepackage{algorithmicx}
\usepackage{algpseudocode}
\usepackage{amsfonts}
\usepackage{amsmath}
\usepackage{bm}
\usepackage{caption}
\usepackage{color}
\usepackage{float}
\usepackage{graphicx}
\usepackage{subfig}
\usepackage{url}
\usepackage{wrapfig}

% Plotting
\usepackage{siunitx}
\usepackage{tikz}
\usepackage{pgfplots}
\pgfplotsset{compat=newest}
\usepgfplotslibrary{units}
\sisetup{round-mode=places, round-precision=2 }

% Code blocks
\usepackage{listings}
\lstset{breaklines=true}

% Title page
\university{Vilniaus universitetas}
\faculty{Matematikos ir informatikos fakultetas}
\institute{Informatikos institutas}
\department{Informatikos katedra}
\papertype{Baigiamasis bakalauro darbas}
\title{Naujausių automatizuoto slaptažodžių parinkimo metodų palyginimas}
\titleineng{Comparison of the latest automated password guessing techniques}
\status{4 kurso 1 grupės studentas}
\author{Gediminas Valys}
\supervisor{prof. dr. Igoris Belovas}
\reviewer{doc. dr. Vardauskas Pavardauskas}
\date{Vilnius \\ 2022}

\setmainfont{Palemonas}
\bibliography{bibliografija} 

\begin{document}
\maketitle

\tableofcontents

\sectionnonum{Sąvokų apibrėžimai}
Sutartinių ženklų, simbolių, vienetų ir terminų sutrumpinimų sąrašas (jeigu
ženklų, simbolių, vienetų ir terminų bendras skaičius didesnis nei 10 ir
kiekvienas iš jų tekste kartojasi daugiau nei 3 kartus).
% Sudeti daznai naudotas savokas cia kad nesikartotu tekste

\section{Santrauka}
Santrauka lietuviškai.

\section{Summary}
Summary in English.

\section{Įvadas}
% Autentifikacija slaptažodžiu
Vartotojo autentikacija slaptažodžiu yra vienas iš pagrindinių kibernetinio 
saugumo metodų.
% Kodėl yra poreikis slaptažodį nustatyti
Baudžiamųjų bylų tyrimuose, siekiant nustatyti kaltę, skaitmeninės informacijos 
specialistams pavedama užduotis nuskaityti kaltinamojo kompiuterių, mobiliųjų 
įrenginių informaciją ir atlikti informacijos analizę, atsakyti į užduotyje 
pateiktus klausimus - jei įrenginys apsaugotas skaitmeninės informacijos 
šifravimo priemonėmis (angl. \textquote{Full Disk Encryption}), taip pat svarbu 
nustatyti prisijungimo/apsaugos/iššifravimo slaptažodį - kitaip informacijos 
analizės atlikti nėra galimybės.
% Plačiau apie probleminę sritį
Tokiose atvejuose, kai reika iššifruoti skaitmeninę informacijos laikmeną arba 
prisijungti prie vartotojo internetinės paskyros nežinant apsaugos slaptažodį, 
pasitelkiama dabar naudojamais slaptažodžio parinkimo metodais: atspėjimas 
brutalios jėgos būdu, naudojant įvairias nutekintas slaptažodžių duomenų bazes. 
Aukščiau minėti metodai yra ribotų galimybių - brutalios jėgos būdu ilgesnius 
nei 6 simbolių ilgio slaptažodžius nustatyti praktiškai neįmanoma dėl 
eksponentiškai išaugusio sudėtingumo, o slaptažodžių sąrašuose tikimybė, kad yra 
ieškomas slaptažodis, maža.
% Tikslas - palyginti du naujausius automatizuoto slaptažodžio parinkimo 
% metodus, kurie gali ir sudėtingesnius slaptažodžius lengvai parinkti
Naujausi automatizuoto slaptažodžių parinkimo metodai, pagrįsti mašininiu 
mokymusi metodais ir galintys identifikuoti slaptažodžių struktūrą, žmogaus 
akimi nematomas slaptažodžių ypatybes, turi potencialo parinkti ir unikalius, 
ilgesnius, nei 6 simbolių ilgio, slaptažodžius, kurių slaptažodžių sąrašuose 
nėra. Šio darbo tikslas yra palyginti tokius metodus - įvertinant jų efektyvumą, 
privalumus ir trūkumus.
% Kaip aš tai atliksiu - įvardinti uždavinius, tyrimo metodologija
Iškelto tikslo pasiekimui šiame darbe bus atliekami eksperimentai su aukščiau 
minėtais slaptažodžių parinkimo metodais, kuriose siekiama parinkti kuo didesnį 
kiekį iš anksto nežinomų slaptažodžių nutekintuose slaptažodžių duomenų bazėse. 
Apmokius slaptažodžių parinkimo modelį su pasirinktomis atvirai prieinamomis 
nutekintomis slaptažodžių duomenų bazėmis, bus fiksuojamas parinktų 
slaptažodžių, persidengiančių su nežinomų slaptažodžių aibe, kiekis. Išbandžius 
skirtingus modelio apmokymo slaptažodžių duomenų bazes, užfiksavus ir įvertinus 
pasiektus rezultatus, vedamos išvados.

\section{Literatūros apžvalga}
\subsection{Bendros sąvokos}
Šioje dalyje siekiama apibūdinti bei įsigilinti į vėlesniuose skyriuose 
nagrinėjamas temas ir idėjas, kuriomis yra paremti slaptažodžių parinkimo 
metodai, pasirinkti sprendimai.

\subsubsubsection{Maišos funkcija ir kontrolinė suma} \label{sec:hashing}
Slaptažodžių parinkimui bei saugojimui yra pasitelkiama kriptografinėmis maišos 
funkcijomis (angl. \textquote{Hash Function}) ir jų savybėmis - kriptografijoje 
naudojamas skaičiavimas, kuris iš tam tikros įvesties grąžina unikalę vertę, 
vadinama kontroline suma (angl. \textquote{Hash}), kuri dalinai arba pilnai 
pasikeičia atspindint menkiausią pasikeitimą įvesties duomenyse. Pavyzdžiui, 
maišos funkcija \textquote{MD5} (angl. \textquote{Message Digest Algorithm 5}) 
įvestiems duomenims \textquote{duomenys} grąžina išvestį 
\textquote{df4111da0615399abad70223d1d74b9b}; pakeitus įvesties duomenų vieną 
simbolį - įvesties duomenys yra \textquote{duomenis} - atitinkama 
\textquote{MD5} maišos funkcijos kontrolinė suma bus 
\textquote{a9d5bc509e33c1d98dca1ba876c9045b}.

Paminėtina, kad iš kontrolinės sumos įvesties duomenų atkurti paprastai 
neįmanoma, tačiau tarp skirtingų įvesties duomenų gali būti ir kolizijų - 
(dalinai) sutampančių kontrolinių sumų, pagal kurias galima (dalinai) atspėti 
pradinius duomenis. Siekiant išvengti kolizijų, rekomenduojama naudoti stiprias 
kriptografines maišos funkcijas, kurių kolizijų tikimybės yra labai mažos, 
praktiškai neįmanomos. Kaip pavyzdį paėmus \textquote{SHA-256} kriptografinę 
maišos funkciją (angl. \textquote{Secure Hash Algorithm 256 Bit}), kurios 
kontrolinė suma yra 64 simbolių (32 baitų, 256 bitų) ilgio, sudaryta iš 
skaitmenų ir lotyniškos abėcėlės raidžių, tikimybė, kad kontrolinė suma 
nepasikeistų net su mažiausiu pasikeitimu pradiniuose duomenyse (pvz. pakeitus 
vieną bitą) yra $\frac{1}{2^{256}}$ - beveik lygi nuliui. Palyginus su anksčiau 
minėtu \textquote{MD5} kriptografinės maišos funkcijos algoritmu, kurio 
kontrolinės sumos yra 32 simbolių (16 baitų, 128 bitų) ilgio, tikimybė, kad bus 
vienoda kontrolinė suma - $\frac{1}{2^{128}}$. Praktikoje jau buvo rastos 
kolizijos su \textquote{MD5} algoritmu.
% TODO: čia būtų gerai įdėti nuorodą arba šaltinį, kaip ir kada buvo rastos

Maišos funkcijos yra naudojamos suskaičiuoti slaptažodžių kontrolines sumas - 
dažnai prie slaptažodžių pridedama papildoma informacija, kuri gali būti saugoma 
duomenų bazėse arba šifruotų laikmenų antraštėse. Prieš skaičiuojant 
slaptažodžio kontrolinę sumą, gera praktika yra prie slaptažodžio pridėti 
papildomą informaciją, kaip pvz. vadinamoji \textquote{druska} (angl. 
\textquote{salt}), skirta apsunkinti tam tikras slaptažodžių parinkimo atakas 
(naudojant iš anksto suskaičiuotus slaptažodžių kontrolinių sumų sąrašus, angl. 
\textquote{Rainbow Tables}). \textquote{Druska} dažnai būna atsitiktiniu būdu
parinkta simbolių, raidžių ir skaičių seka, kuri yra saugoma kartu su galutine 
slaptažodžio ir \textquote{druskos} kontroline suma. Kai vartotojas prisijungia 
prie internetinės svetainės arba įveda kompiuterio skaitmeninės laikmenos 
iššifravimo slaptažodį, įvestas slaptažodis yra sujungiamas su 
\textquote{druska} ir palyginamas su turima kontroline suma.

\subsubsubsection{Slaptažodžių parinkimo procesas}
Slaptažodžių parinkimas šnekamojoje kalboje taip pat vadinamas 
\textquote{slaptažodžių laužymu} - procesas, kai turimą nežinomo slaptažodžio 
kontrolinę sumą yra bandoma atspėti parenkant tikrąjį, maišos funkcijos 
neapdorotą slaptažodį. Turint slaptažodį, kurį norima patikrinti, slaptažodis 
yra apdorojimas maišos funkcija ir palyginama išvesta kontrolinė suma su turima 
kontroline suma - jei kontrolinės sumos sutampa, slaptažodis buvo 
atspėtas/parinktas (\textquote{nulaužtas}).

\subsubsubsection{Mašininis mokymasis}
Mašininis mokymasis, tame tarpe dirbtinis intelektas ir neuroniniai tinklai, 
paprastai naudojami duomenų analizei, prognozavimui, klasifikavimui ir naujų 
duomenų generavimui. Konkretus pritaikymas šiai technologijai yra didiesiems 
duomenims (angl. \textquote{Big Data}) - masinis kiekis duomenų gali būti 
naudojami prognozuoti rinkos tendencijas, žmonių elgesį, transformuoti ir 
generuoti kitus duomenis apmokius kompiuterinį modelį įžvelgti paprastai 
nematomus parametrus, požymius duomenyse. Mašininio mokymosi modelis - parametrų 
ir duomenų rinkinys, kurie yra nustatomi, koreguojami ir/ar surenkami modelio 
apmokymo procese. Išskiriami du pagrindiniai mašininio mokymosi proceso tipai:
% TODO įdėti panaudojimo atvejus?
\begin{itemize}
  \item su mokytoju (angl. \textquote{Supervised Learning}) - mašininio mokymosi 
    algoritmui yra pateikiami iš anksto suklasifikuoti, apdoroti duomenys, 
    kuriuose algoritmas turi identifikuoti tokius reikšmingus požymius, kad 
    modelio priimami sprendimai sutaptų su mokytojo nustatytais sprendimais.
    Algoritmo pradiniai duomenyse kiekvienai reikšmei yra priskirta tam tikra 
    žymė (asociacija), ir algoritmas duomenyse esamus požymius (pvz. forma, 
    struktūra, pasiskirstymą, spalvą ir pan.) susieja su priskirta žyme. 
    Mokymasis su mokytoju toliau gali būti skirstomi į du tipus:
    \begin{itemize}
      \item klasifikacija - priskirti duomenis prie tam tikros klasės, grupės;
      \item regresija - prognozuoti duomenis, priklausančius nuo kitų duomenų.
    \end{itemize}
  \item be mokytojo (angl. \textquote{Unsupervised Learning}) - mašininio 
    mokymosi algoritmui yra pateikiami \textquote{žali} duomenys, t.y. ne 
    klasifikuoti ar kitaip apdoroti, ir modelis turi pats identifikuoti, 
    išskirti duomenyse reikšmingus požymius, ryšius. Mokymasis be mokytojo 
    toliau gali būti skaidomas į du tipus:
    \begin{itemize}
      \item grupavimas - reikšmių padalijimas į grupes, kurios turi panašius 
        požymius;
      \item asociacija - taisyklėmis pagrįstas mokymasis, kuriame nustatomi 
        tarpusavio ryšiai tarp reikšmių.
    \end{itemize}
\end{itemize}

\subsection{Brutalios jėgos metodas slaptažodžio parinkimui} 
\label{sec:bruteforce}
Jei vartotojas turėtų iki 6 simbolių ilgio slaptažodį, tai programa ar 
kompiuteris galėtų sugeneruoti visus įmanomus 1-6 simbolių ilgio slaptažodžius, 
su visomis raidžių, skaičių ir simbolių (mišinių) permutacijomis. Tokiu būdu yra 
beveik garantuota parinkti sutampantį su vartotojo slaptažodį brutalios jėgos 
būdu (angl. \textquote{Brute Force}) per palyginti trumpą laiką. Pavyzdžiui, su
mano 9 metų senumo nešiojamuoju kompiuteriu, kuriame yra Intel i5-2520M 
procesorius ir Python programine kalba parašytu scenarijumi, pateiktame 
\ref{fig:kombinacijos} pav., sugeneravau visas 6 simbolių kombinacijas (raidžių 
aibę sudaro lotyniška abėcėlė) per X minutes, kurios sudaro Y eilutes ir užima Z 
GB vietos skaitmeninėje laikmenoje (dar bus užpildyta).

\begin{figure}
  \begin{center}
    \lstinputlisting[
      language=Python,
      numbers=left
    ]{permutations.py}
    \caption{Programinis kodas sugeneruoti visas 6 simbolių kombinacijas.}
  \end{center}
  \label{fig:kombinacijos}
\end{figure}

\subsection{Slaptažodžių sąrašai slaptažodžio parinkimui} \label{sec:wordlists}
Internete paplitusios vartotojų slaptažodžių nutekintos duomenų bazės yra 
tinkamos slaptažodžių parinkimui. Vartotojai dažnai naudoja tuos pačius 
slaptažodžius daugeliui savo kuriamų internetinių svetainių paskyrų, tačiau 
nebūtinai visi šių internetinių svetainių kūrėjai/aptarnautojai laikosi
informacinės saugos taisyklių ir gerųjų praktikų, tokių, kaip:
\begin{enumerate}
  \item nesaugoti slaptažodžius paprastu tekstu (angl. \textquote{Plain Text});
  \item apdoroti slaptažodžius stipriomis kriptografinėmis maišos funkcijomis;
  \item su \textquote{druska} apsaugoti slaptažodžius nuo dažniausiai naudojamų 
        atakų, kaip pateikta \ref{sec:hashing} skyriuje.
\end{enumerate}

Kai veikėjas (asmuo, programa) įsiveržia į tokias internetines svetaines ir 
nutekina prisiregistravusių vartotojų asmeninius duomenis, tame tarpe ir 
prisijungimo slaptažodžius, jie būna parduodami internete arba patalpinami 
nemokamai laisvai prieinamuose internetinėse svetainėse.
% TODO Gal įdėti pavyzdžių kokių nors populiariausių tokių wordlistų?
Naudojantis šias duomenų bazes su vartotojų prisijungimo informacija, kiekvienas 
slaptažodis yra apdorojimas maišos funkcija ir jos išvesta kontrolinė suma 
palyginama su turimomis nežinomomis slaptažodžių kontrolinių sumomis.

\subsection{Slaptažodžių kontrolinių sumų sąrašai slaptažodžio parinkimui} 
\label{sec:rainbowtables}
Slaptažodžių apdorojimui į kontrolines sumas, naudojant kriptografines maišos 
funkcijas, tokiais dideliais kiekiais, kaip nutekintų slaptažodžių duomenų 
bazių, reikalingi dideli kompiuteriniai ištekliai. Siekiant minimizuoti laiką, 
skirtą slaptažodžio parinkimui, nutekintos slaptažodžių duomenų bazės (ar jų 
mišinio) slaptažodžiai yra iš anksto apdorojami kriptografinės maišos funkcijos 
(toks metodas vadinamas angl. \textquote{Rainbow Tables}). Tokiu būdu, 
slaptažodžio užpuolikui, siekiančiam nustatyti tikrąją slaptažodžio reikšmę, 
reikia tik palyginti kontrolines sumas ir surasti, jei tokia yra, sutampančią. 
Kaip minėta \ref{sec:hashing} skyriuje, tokio tipo ataka netinkama 
slaptažodžiams, kurie yra apsaugoti su \textquote{druska}, kadangi kontrolinės 
sumos skirtųsi net ir jeigu tikra slaptažodžio reikšmė sutampa.

Paminėtina, kad naudojantis tokiais sugeneruotais sąrašais, kaip pateikta 
\ref{sec:bruteforce}, \ref{sec:wordlists} ir \ref{sec:rainbowtables} skyriuose, 
vartotojas galėtų paskirstyti skaičiavimus (slaptažodžio parinkimą) kelioms 
procesoriaus gijoms, vaizdo plokštėms ar kompiuteriams slaptažodžių sąrašą 
padalinant į dalis, siekiant pagreitinti slaptažodžio parinkimą.

\subsection{Slaptažodžių transformavimas naudojant taisykles}
Atvirojo kodo programinė įranga \textquote{hashcat}\footnotemark, skirta 
slaptažodžių atstatymui - nustato tikrąjį slaptažodį taikant brutalios jėgos ir 
kt. metodus turimo slaptažodžio kontrolinei sumai - palaiko teksto eilučių 
(slaptažodžių) transformavimą taisyklėmis. Taisyklės yra tekstiniai failai, 
kuriose yra pateiktos instrukcijos kaip transformuoti 

\footnotetext{
  Programinės įrangos internetinė svetainė: \url{https://hashcat.net/hashcat/}.
}

\subsection{Ankstesni darbai}
Automatizuoto slaptažodžio parinkimo metodų srityje yra atlikti moksliniai 
tyrimai, nagrinėjantys galimybes pritaikyti ir integruoti mašininį mokymąsi bei 
kitus metodus, kad slaptažodžio parinkimo procesas sudėtingesniems, nei 1-6 
simbolių ilgio slaptažodžiams, vyktų lengviau (greičiau). Šie metodai 
automatizuotu būdu generuoja taisykles iš pradinių duomenų (paprastai naudojamos 
nutekintos slaptažodžių duomenų bazės), ir pagal sugeneruotas taisykles - naujus 
slaptažodžius. Žemiau yra paminėti keli tokie metodai:
\begin{enumerate}
  \item \textquote{PassGAN} - paremtas mašininio mokymosi generatyviniais 
    konkurenciniais tinklais;
  \item \textquote{PCFG} - paremtas tikimybine bekontekste gramatika;
  \item \textquote{GenPASS} - pagrįstas mašininio mokymosi pasikartojančiais 
    neuroniniais tinklais ir tikimybiniais gramatikos taisyklių rinkiniais;
  \item \textquote{TG-SPSR} - slaptažodžių parinkimo strategija, pagrįsta 
    tikimybiniais gramatikos taisyklių rinkiniais ir Markovo grandinių 
    modeliais.
\end{enumerate}
% Ne tik pasakyti ką atliko kiti autoriai, bet palyginti jų darbus, pateikti tų 
% darbų pliusus ir minusus

% Kaip veikia šitas algoritmas, pateikti pavyzdžių (įdėti transformacijų medį)

% Kokie šio algoritmo pagrindiniai požymiai, pastebėti trūkumai ir minusai
% TODO parasyti apie kitus algoritmus
% idet kas yra masininis mokymasis ir pan, viska aprasyt kad belekas suprastu,
% isivaizduot kad bet kokios srities zmogus skaito, ir aprasyt taip, kad 
% suprastu

\subsection{\textquote{PassGAN} metodas}
Šioje dalyje apibendrinsiu generatyvinius konkurencinius tinklus, pagal kuriuos 
yra pagrįstas \textquote{PassGAN} metodas, bei pateiksiu esminius automatizuoto 
slaptažodžių parinkimo realizacijos bruožus.

% Kas yra GANai
% Pritaikymas slaptažodžių parinkimo srityje
% Kaip algoritmas veikia
\subsubsection{Apie generatyvinius konkurencinius tinklus}
Generatyviniai konkurenciniai tinklai (angl. \textquote{Generative Adversarial 
Networks}, \textquote{GAN}) yra generatyvinis modeliavimas naudojant gilaus 
mašininio mokymosi metodus. Generatyvinis modeliavimas yra mašininio mokymosi 
metodas be mokytojo, kuriame modelis yra apmokamas generuoti tokias reikšmes, 
kurios galėtų tikėtinai būti iš apmokymo duomenų rinkinio. Generatyviniame 
konkurenciniame tinkle yra dvi dalys - generavimo modelis (angl. 
\textquote{Generator Model}), kuris yra apmokamas generuoti naujas reikšmes, ir 
diskriminatoriaus modelis (angl. \textquote{Discriminator model}), kuris 
klasifikuoja generuojamas reikšmes kaip tikras arba netikras. Generatyvinių 
konkurencinių tinklų pagrindas yra teoriškas žaidybinis scenarijus, kuriame 
minėti du modeliai konkuruoja ir varžosi - vienas modelis generuoja reikšmes, 
kitas bando atskirti jas nuo generatoriaus ir apmokymo duomenų 
\cite{Goodfellow-et-al-2016}. Generavimo modelis yra apmokamas tol, kol 
generuoja tokias įtikinamas reikšmes, kurias diskriminatoriaus modelis negali 
atskirti nuo realių (apmokymo) reikšmių.

Pagrindinis skirtumas tarp diskriminuojančio ir generatyvinio modeliavimo yra 
tai, diskriminacinis modelis klasifikuoja - turi būti priimtas sprendimas, 
kuriai klasei priklauso tam tikra reikšmė, o generatyvinis - generuoja arba 
sukuria naujas reikšmes pagal pradinius duomenis, kurie galėtų tikėtinai būti 
pradiniame duomenų rinkinyje. Kitaip tariant, generatyvinis modelis gali būti 
apmokytas pradinių duomenų pasiskirstymo ir gali generuoti naujas reikšmes, 
kurios galėtų tikėtinai būti priskirtos prie pradinių duomenų \cite{Bishop07}.

Generavimo modeliui yra pateikiama kaip įvestis nustatyto ilgio vektorius iš 
atsitiktinai paskirstytos duomenų aibės (angl. \textquote{Gaussian 
distribution}). Apmokymo metu tam tikriems (paprastai nematomiems) požymiams iš 
šios duomenų aibės yra priskiriama svarba, t.y. generuojamos reikšmės artėja 
prie apmokymo duomenų aibės - generatorius sugeneruoja grupę reikšmių, kurios, 
kartu su apmokymo duomenimis, yra perduodami diskriminatoriui klasifikuoti kaip 
tikros arba netikros. Tokiu būdu diskriminatoriaus parametrai yra tobulinami 
tiksliau klasifikuoti tikras/netikras reikšmes, o generatorius - kiek 
sugeneruotos reikšmės galėjo apgauti/įtikinti diskriminatorių (diskriminatoriaus 
reikšmės, nustatytos kaip tikros arba netikros, atskleidžia reikalingą 
informaciją apie apmokymo duomenų aibę generatoriui, kuris gali ją naudoti 
tiksliau modeliuoti apmokymo duomenų pasiskirstymą). Po modelio apmokymo, minėta 
duomenų aibė turėtų atitikti apmokymo duomenų aibę, t.y. diskriminatorius negali 
su didesniu nei apie 50 \% tikslumu (pasitikėjimu, arba angl. 
\textquote{Confidence}) atskirti apmokymo duomenų aibės reikšmes nuo generuojamų 
reikšmių.
\subsubsection{\textquote{PassGAN} metodo požymiai}
\textquote{PassGAN} metodui, kuriame generatyvinis konkurencinis tinklas yra 
pritaikytas slaptažodžių generavimui, yra reikalingi palyginti 


\cite{DBLP:journals/corr/abs-1709-00440}

\subsection{\textquote{PCFG} Algoritmas}
\subsubsection{Natūralios kalbos apdorojimas}
Parašyti apie NLP sritį, kaip ji siejasi su gramatikomis ir pan.

\subsubsection{Apie tikimybines, bekontekstes gramatikas}
\textquote{PCFG} (angl. \textquote{Probabilistic Context Free Grammar}) - 
tikimybinis gramatikos taisyklių rinkinys, nurodantis, kaip kažkurį tai 
simbolį-reikšmę galima transformuoti į kitą simbolį. \textquote{PCFG} algoritmas 
yra pagrįstas bekontekste gramatika (angl. \textquote{Context Free Grammar}) - 
notacija, skirtą apibrėžti tam tikros formalios kalbos sintaksę, pvz. 
programinio kodo šaltinių failų analizei ir apdorojimui į simbolių grupių 
leksinį medį (kaip simbolių grupės yra susijusios su viena kita, ką jos 
reiškia). Tikimybinė bekontekstė gramatika (angl. \textquote{Probabilistic 
Context Free Grammar}) yra bekontekstė gramatika, kurioje kiekvienai simbolių 
grupei yra priskirta tikimybė. Tikimybinė bekontekstė gramatika yra kilusi iš 
kompiuterinės lingvistikos srities, skirta analizuoti ir modeliuoti natūralią 
kalbą - simbolių grupių (pasikartojimo) tikimybės gali būti naudojamos kaip 
parametrai mašininio mokymosi modelyje, tačiau tikimybių reikšmės ypač priklauso 
nuo duomenų (kiekio, įvairovės), iš kurių jos yra išvestos.

\subsubsection{Pritaikymas slaptažodžių parinkimui}
Tikimybinės bekontekstės gramatikos metodą galima pritaikyti ir slaptažodžių 
gramatikos (formalios kalbos notacijos) nustatymui, ir, pagal šią gramatiką, 
parinkti naujus slaptažodžius. Naujų slaptažodžių parinkimui yra saugomos 
simbolių grupių reikšmės - slaptažodžiai yra dalinami ir sugrupuojami į 
sujungtas skaitmenų, raidžių ir specialiųjų simbolių grupes, kurios yra 
įstatomos į slaptažodžio gramatiką parenkant naujus slaptažodžius.

\subsubsection{Algoritmas}
Slaptažodžių parinkimo strategijoje, \textquote{PCFG} algoritmas kiekvienam 
slaptažodžiui iš apmokymo duomenų rinkinio pirmiausia išveda gramatiką, t.y. 
simbolių grupės yra sugrupuojamos pagal tipą: skaitmenys, raidės, specialūs 
simboliai ir kartu užfiksuojami su konkrečių simbolių sekų bei pačios gramatikos 
pasikartojimų skaičiumi medžio pagrindo duomenų struktūroje (paprastai binarinis 
medis, angl. \textquote{Binary Tree}). Kai visi slaptažodžiai yra apdoroti į jų 
atitinkamas gramatikas, paskaičiuojami gramatikų ir simbolių grupių 
pasikartojimo dažniai - kiek kartų gramatika ir ją sudarančios simbolių grupių 
reikšmės pasikartoja visoje apmokymo duomenų aibėje. Minėti dažniai yra 
naudojami parenkant naujus slaptažodžius - pirmiausia parenkami slaptažodžiai, 
kurių gramatika ir gramatikoje esančių simbolių grupių dažniai yra didžiausi, 
tokiu būdu slaptažodžiai su aukščiausia tikimybe yra parenkami pirmi. Kitaip 
tariant, slaptažodžiai, kurių forma, struktūra ir skaitmenų, raidžių, specialių 
simbolių junginių grupės pasikartoja daugiausiai, sudaro šio metodo pirmąsias 
išvestis.
Viena gramatika \textquote{PCFG} algoritme gali būti panaudota kelis kartus 
parenkant slaptažodžius. \textquote{PCFG} algoritme yra išskiriami du tipai 
slaptažodžių - preliminarūs (angl. \textquote{Pre-terminal}) ir galutiniai 
(angl. \textquote{Terminal}). Kiekvienas preliminarus slaptažodis yra parenkamas 
atsižvelgiant į jo ašies reikšmę, t.y. simbolių grupės, kuri buvo 
pakeista/transformuota parenkant šį slaptažodį, indeksas (pirminė ašies reikšmė 
parenkant pirmuosius preliminarius slaptažodžius yra lygi nuliui). Galutinių 
slaptažodžių parinkimo ir preliminarių slaptažodžių transformavimo taisyklės yra 
tokios:
\begin{enumerate}
  \item preliminariame slaptažodyje simbolių grupė gali būti keičiama tik jei 
    preliminaraus slaptažodžio ašies vertė yra mažesnė už grupės indeksą 
    slaptažodžio gramatikoje;
  \item preliminaraus slaptažodžio nauja ašis yra slaptažodyje pakeistos 
    simbolių grupės indeksas, skaičiuojant nuo 0;
  \item iš preliminaraus slaptažodžio yra parenkamas galutinis slaptažodis kai 
    yra pakeičiama viena ir tik viena simbolių grupė.
\end{enumerate}

Vadovaujantis aukščiau minėtomis taisyklėmis yra parenkami nauji galutiniai ir 
preliminarūs slaptažodžiai su transformuotomis simbolių grupėmis. Galutinis 
slaptažodis algoritmo yra grąžinamas, kai nebėra daugiau transformacijų, kurios 
galėtų būti su juo atliktos, o nauji preliminarūs slaptažodžiai, kuriems jau 
buvo atliktos transformacijos, yra grąžinami į algoritmo sąrašą tolesnėms 
transformacijoms.

\subsubsection{Pastebėjimai}
Algoritmas pabaigą pasiekia tada, kai gramatikos arba unikalios simbolių grupės 
pasibaigia ir/arba naujiems slaptažodžiams parinkti trūksta duomenų - raidžių, 
skaitmenų ar specialių simbolių junginių. Parenkamų slaptažodžių bei gramatikų 
ir simbolių grupių kiekiai priklauso nuo apmokymo duomenų rinkinio dydžio - yra 
svarbu pateikti ne didelį duomenų kiekį, bet duomenis, kuriose yra daug 
pasikartojančių reikšmių. Šios pasikartojančios reikšmės yra naudojamos 
tikimybių skaičiavimams ir turi įtaką, kokie slaptažodžiai bus parenkami 
slaptažodžių parinkimo proceso pradžioje.
Algoritmas slaptažodžius parinks tik tokius, kurių gramatikos buvo užfiksuotos 
apmokymui pateiktų slaptažodžių aibėje - 

% TODO Giliau panagrineti problema, duoti konkreciu pavyzdziu pvz is Passware, 
% sifruota particija su LUKS

% Naudoti kita PCFG implementacija, o ne mano, geresni skaiciai gausis tada
% galima rasyti, kad tipo realizacija paremta pagal tokie ir toki projekta,
% det isnasa i konkrecia nuoroda

\section{Pagrindinė tiriamoji dalis}
Pagrindinėje tiriamojoje dalyje aptariama ir pagrindžiama tyrimo metodika;
pagal atitinkamas darbo dalis, nuosekliai, panaudojant lyginamosios analizės,
klasifikacijos, sisteminimo metodus bei apibendrinimus, dėstoma sukaupta ir
išanalizuota medžiaga. 

\subsection{Etika}
Šis darbas buvo atliktas pagal Europos elgesio kodeksą mokslinių tyrimų etikos 
klausimais
(angl. \textquote{European Code of Conduct for Research Integrity}), paruoštą 
\textquote{European Federation of Academies of Sciences and Humanities} (ALLEA) 
organizacijos. Kadangi nutekintos duomenų bazės yra sudarytos iš galimų 
vartotojų asmeninės informacijos (pvz. vardas, pavardė, elektroninis pašto 
adresas), visa informacija buvo saugiai laikoma ir tvarkoma. Asmens duomenų 
atskleidimo galimybės buvo mažinamos įgyvendinant griežtas saugumo priemones, ir 
nutekinti slaptažodžiai nebuvo testuojami su tikromis internetinių paslaugų 
vartotojų paskyromis. Darbe atskleidžiami tik patys dažniausi slaptažodžiai, jų 
sudėtingumas ir struktūra. Slaptažodžiai, kurie galėtų būti naudojami 
identifikuoti vartotojus, nebuvo atskleisti. Vartotojų asmeninė informacija yra 
nepateikta.

\subsection{Nutekintų slaptažodžių duomenų bazių analizė} \label{sec:db-analize}
Šiame darbe naudojamos dvi nutekintų slaptažodžių duomenų bazės, atvirai 
prieinamos viešais šaltiniais - RockYou ir . Duomenų bazę (rinkinį) C sudaro 
slaptažodžių SHA-1 maišos funkcijos kontrolinės sumos. Dešimt dažniausiai 
pasikartojančių slaptažodžių duomenų bazėje A yra pateikti \ref{10-dazn-slapt-a} 
lentelėje, o duomenų bazėje B - \ref{10-dazn-slapt-b} lentelėje.

% $ sort DB.txt | uniq -c | sort -nr | head -n 10
\begin{table}[hb]
\centering
\begin{tabular}{|c|c|}
  \hline
  Kiekis & Slaptažodis \\
  \hline
  661 & 123456 \\
  363 & aaaaaa \\
  313 & 123456789 \\
  187 & qwerty \\
  125 & vasara \\
  122 & pavasaris \\
  122 & kaktusas \\
  119 & slaptazodis \\
  116 & katinas \\
  107 & 12345678 \\
  \hline
\end{tabular}
\caption{Duomenų rinkinio A 10 dažniausiai pasikartojančių slaptažodžių}
\label{10-dazn-slapt-a}
\end{table}

\ref{10-dazn-slapt-a} lentelėje pateikti slaptažodžiai daugiausia sudaryti iš 
bendrinių daiktavardžių, paprastų skaičių ir raidžių sekų.

\begin{table}[ht]
\centering
\begin{tabular}{|c|c|}
  \hline
  Kiekis & Slaptažodis \\
  \hline
  2252 & x4ivygA51F \\
  1341 & 123456789 \\
  1302 & h54rsjrF5J46788998 \\
  993 & 64t3zaWonZ \\
  917 & 2u8Qujf5eE \\
  896 & slaptazodis \\
  893 & 88ggOx8ouG \\
  828 & lopas123 \\
  823 & 5ogs6d4QnbA \\
  761 & Y52flzfq8V \\
  \hline
\end{tabular}
\caption{Duomenų rinkinio B 10 dažniausiai pasikartojančių slaptažodžių}
\label{10-dazn-slapt-b}
\end{table}

Kaip yra pavaizduota \ref{10-dazn-slapt-b} lentelėje, kitaip nei 
\ref{10-dazn-slapt-a} lentelėje, dauguma slaptažodžių yra sudėtingesni skaičių 
ir simbolių sekų mišiniai, kuriuose bendriniai daiktavardžiai yra retesni.

\begin{figure}[!ht]
  \begin{center}
    \begin{tikzpicture}
      \begin{axis}[
        xlabel={Sugeneruoti slaptažodžiai},
        ylabel={Sutampantys slaptažodžiai},
        x unit={vnt.},
        y unit={vnt.},
        grid=major,
        grid style={dashed,gray!30},
        width=10cm,
        height=7cm,
        legend style={at={(0.5,-0.2)}, anchor=north}
      ]
      \addplot[color=red, mark=*]
        table[x=column 1, y=column 2, col sep=comma]{table1.csv};
      \addplot[color=blue, mark=*]
        table[x=column 1, y=column 2, col sep=comma]{table2.csv};
      \legend{Case 1,Case 2}
      \end{axis}
    \end{tikzpicture}
    \ref{plot-1}
    \caption{Plot.}
  \end{center}
\end{figure}

\subsection{Eksperimentuose taikoma strategija}
Su pasirinktu automatizuoto slaptažodžių parinkimo metodu bus apmokytas modelis 
su 80\% duomenų nuo visos pasirinktos nutekintos slaptažodžių duomenų bazės 
įrašų aibės, likusieji 20\% bus naudojami testavimui. Nutekintoms slaptažodžių 
duomenų bazėms, kurios minėtos ir analizuotos \ref{sec:db-analize} skyriuje, bus 
naudojamos modelių apmokymui. Siekiant įvertinti automatizuoto slaptažodžio 
parinkimo metodo efektyvumą, bus generuojami tarp $10^{6}$ ir $10^{10}$ 
slaptažodžių, kurie bus testuojami su 20\% įrašų iš apmokymui naudojamos duomenų 
bazės testavimo duomenų aibės, ir su kita, anksčiau nematyta nutekinta 
slaptažodžių duomenų baze. Slaptažodžių kiekis, sutampančių su 20\% testavimo 
duomenų aibės slaptažodžiais ir su kitoje nutekintoje duomenų bazėje esančiais 
slaptažodžiais, bus fiksuojamas kiekvienai sugeneruotų slaptažodžių rinkiniui.

\subsection{\textquote{PassGAN} eksperimentų serija}

\subsection{\textquote{PCFG} eksperimentų serija}

\sectionnonum{Išvados}
Išvadose ir pasiūlymuose, nekartojant atskirų dalių apibendrinimų,
suformuluojamos svarbiausios darbo išvados, rekomendacijos bei pasiūlymai.
% Ar atlikta studija sėkmingai išsprendė užsibrėžtą tikslą?
% Kokie apribojimai buvo taikomi, kurie trukdė atlikti tyrimą?
% Ar gauti rezultatai yra patikimi?
% Ar buvo atliktas palyginimas su kitų autorių analogiškais rezultatais?
% Kaip rezultatai gali būti panaudoti ateityje?
% Kokio tipo naują informaciją/sprendimą/tyrimą galima pažymėti?
% Kokias naujas idėjas ir pasiūlymus iškelia gauti rezultatai?

\sectionnonum{Conclusions}
Šiame skyriuje pateikiamos išvados (reziume) anglų kalba.

\printbibliography[heading=bibintoc]

\appendix  % Priedai
% Prieduose gali būti pateikiama pagalbinė, ypač darbo autoriaus savarankiškai
% parengta, medžiaga. Savarankiški priedai gali būti pateikiami kompiuterio
% diskelyje ar kompaktiniame diske. Priedai taip pat vadinami ir numeruojami.
% Tekstas su priedais siejamas nuorodomis (pvz.: \ref{img:mlp}).

% \section{Niauroninio tinklo struktūra}
% \begin{figure}[H]
%     \centering
%     \includegraphics[scale=0.5]{img/MLP}
%     \caption{Paveikslėlio pavyzdys}
%     \label{img:mlp}
% \end{figure}

\section{Eksperimentinio palyginimo rezultatai}

\end{document}
